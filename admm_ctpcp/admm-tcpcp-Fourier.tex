\documentclass[a4paper,12pt]{article}
\usepackage{savesym}
\usepackage{stmaryrd}
\usepackage{amsfonts}
\usepackage{pifont}
\usepackage{bbding}
\usepackage{amssymb}

\usepackage{CJK}
\usepackage{amsmath, amsthm, amssymb, upgreek}
\usepackage{titlesec, titletoc, lastpage}   %%章节标题格式
\usepackage{ifthen, bbm, syntonly, CJKpunct}
\usepackage{graphicx, picins, subfigure}
\usepackage{indentfirst}%%首行缩进
\usepackage{geometry}   %%设置页边距
\usepackage{bm}         %%定义粗斜体表示向量
%\usepackage{calrsfs}    %%花体大写字母,如余集,幂集,\mathrsfs{P}, 很花
\usepackage{wasysym}    %%用\varint得到直立积分号
\usepackage{xcolor}     %%它是 color 宏包的扩充, 预定68 种颜色, 支持 pgf 包
\usepackage{extarrows}  %%长等号, 长箭头
\usepackage{mathrsfs}
\usepackage{mathtools}
\usepackage{graphicx}
\usepackage{subfigure}
\usepackage{float}
\usepackage{array}

\usepackage{caption}                % 这四行插入algorithm代码
\usepackage{algorithm}
\usepackage{algorithmicx}
\usepackage{algpseudocode}     % 这四行插入algorithm代码

\makeatletter
\newcommand{\rmnum}[1]{\romannumeral #1}
\newcommand{\Rmnum}[1]{\expandafter\@slowromancap\romannumeral #1@}
\makeatother

\begin{CJK*}{GBK}{song}\end{CJK*}   %%设置格式命令里会用到中文,所以要在设置之前放一个空的CJK 环境



%==============设置汉字字体命令 开始=============================
\newcommand{\song}{\CJKfamily{song}}    %% 宋体  %\upshape 直立,汉字不要斜体
\newcommand{\kai}{\CJKfamily{kai}}      %% 楷体
\newcommand{\hei}{\CJKfamily{hei}}      %% 黑体
%==============设置汉字字体命令 结束=============================

%==============设置使用汉字字号命令 开始=============================
\newcommand{\chuhao}{\fontsize{42pt}{\baselineskip}\selectfont}
\newcommand{\xiaochuhao}{\fontsize{36pt}{\baselineskip}\selectfont}
\newcommand{\yihao}{\fontsize{28pt}{\baselineskip}\selectfont}
\newcommand{\erhao}{\fontsize{21pt}{\baselineskip}\selectfont}
\newcommand{\xiaoerhao}{\fontsize{18pt}{\baselineskip}\selectfont}
\newcommand{\sanhao}{\fontsize{15.70pt}{\baselineskip}\selectfont} %% 原为15.75pt, 但用yap打印有问题
\newcommand{\sihao}{\fontsize{14pt}{\baselineskip}\selectfont}
\newcommand{\xiaosihao}{\fontsize{12pt}{\baselineskip}\selectfont}
%==============设置使用汉字字号命令 结束=============================

%==============设置定义,定理,证明 开始=============================
\newtheoremstyle{mystyle}{00em}{0em}{\kai}{-0.2em}{\hei\bfseries}{}{0.5em}{}
\theoremstyle{mystyle}
\newtheorem{definition}{\hskip 1.88em\hei Def.\hskip 0.1em}[section]
\newtheorem{theorem}{\hskip 1.88em\hei Theorem\hskip 0.1em}[section]
\newtheorem{example}{\hskip 1.88em\hei Example\hskip 0.1em}[section]
\newtheorem{problem}{\hskip 1.88em\hei Problem\hskip 0.1em}[section]

\newtheorem{proposition}[theorem]{\hskip 1.88em\hei Proposition\hskip 0.1em }
\newtheorem{addition}[theorem]{\hskip 1.88em\hei Addition\hskip 0.1em }
\newtheorem{property}[theorem]{\hskip 1.88em\hei Property\hskip 0.1em }
\newtheorem{lemma}[theorem]{\hskip 1.88em\hei Lemma\hskip 0.1em }
\newtheorem{corollary}[theorem]{\hskip 1.88em\hei Corollary\hskip 0.1em }
\newtheorem{axiom}{\hskip 1.88em\hei Axiom\hskip 0.1em }[section]
%==============设置定义,定理,证明 结束=============================


%==============设置页面大小位置 开始=====================================
\setlength\textwidth{160mm} %设置页心宽度
\setlength\textheight{225mm} %设置页心高度
\setlength\voffset{0mm} %设置页心垂直偏移量
%\setlength\hoffset{0mm} %设置页心水平偏移量
\setlength\oddsidemargin{0mm} %设置奇数页的页心到装订线的距离
\setlength\evensidemargin{0mm} %设置奇数页的页心到装订线的距离
%==============设置页面大小位置 结束=====================================


%==============章节标题样式 开始=====================================
%%\titleformat{\chapter}{\centering}{\erhao\hei 第\CJKnumber{\thechapter}章}{1em}{\erhao\hei}
\titleformat{\section}{\centering}{\textbf{\sanhao\hei\raisebox{1pt}{\S}}\hspace{5pt}\sanhao\hei\thesection\hspace{-12pt}}{1.5em}{\sanhao\hei}
%==============章节标题样式 结束=====================================


%==============设置页眉页脚 开始==============================
\newpagestyle{MyPgsStylex}
{\sethead %%[\thepage\backslash\pageref{LastPage}][][\kai\textsl{\upshape 组合数学讨论班\,读书报告\ \raisebox{1pt}{\S}\hspace{0.2em}\thesection\hspace{0.4em}\sectiontitle}]
{Tianwei Yue}{\kai\textsl{\upshape CMSE 820 HW$07$}}{\thepage\,/\,\pageref{LastPage}}\setfoot[][][]{}{}{}\headrule\setheadrule{0.4pt}}
%奇数页页眉显示节标题,偶数页页眉显示章标题,\headrule\setheadrule{0.4pt}设置页眉的横线宽度0.4pt, 未设置\headrule
%==============设置页眉页脚 结束==============================



%%========================文======== 档======== 开========始========================

\begin{document}
\begin{CJK*}{GBK}{song}
\CJKindent          %%段首缩进设为两个中文字符的宽度
\CJKtilde           %%CJK*环境会吞掉跟在汉字后面的空格,从而使得源文件中的换行不会在相邻汉字之间产生空白.


\pagenumbering{arabic}      %%页码用阿拉伯数字
\pagestyle{MyPgsStylex}   %%使用已定义的页眉页脚

\setlength\baselineskip{24pt}   %设置行间距,对于12pt 来说是1.5倍行距
\setlength{\parskip}{2pt}       %设置段间距

%% 用下面两条命令可以设定displaymath与上下文的间距
\setlength{\abovedisplayskip}{2pt plus 2pt minus 4pt}
\setlength{\belowdisplayskip}{2pt plus 2pt minus 4pt}


%%========================以上可以不动, 基本上够用了, 不够再说=====================================

%% 格式一般要求:
%% 全部用英文标点, 标点后有空格.
%% 公式不要自动折行, 长公式应独占一行或多行.
%% 即使是单独的 a, b 等, 也应用\ $a$, $b$ 方式, 注意公式里的字母一般是斜体
%% 特殊符号如自然对数的底 e, 圆周率 pi 等, 应该用直立方式表示: $\mathrm{e}$, $\uppi$.
%% 页眉上的 yourname 应在第 73 行的开始处修改
%% 目前就想到这些, 有问题可以Q我或在群中讨论




%%========================文======== 档======== 开========始========================
\thispagestyle{empty}   %%首页不要页眉页脚


\bigskip
%%=================================================================================================
\noindent TCPCP problem:
\begin{equation}\label{e1}
\min_{\mathcal{L},\mathcal{S}}\|\mathcal{L}\|_*+\lambda \|\mathcal{S}\|_1, s.t. P_{\Omega}(\mathcal{S+L})=g.
\end{equation}
where $g=P_{\Omega}\mathcal{X}_0$ is the sampled data. 

Define $\mathcal{G}=\mathcal{F}_3P_{\Omega}\mathcal{F}_3^{-1}$, where $\mathcal{F}_3$ and $\mathcal{F}_3^{-1}$ are the operators representing the Fourier and inverse Fourier transform along the third dimension of tensors. Then we can rewrite the constraint into $\mathcal{G}\mathcal{(\hat{S}+\hat{L})}={\hat{g}}$. We want to solve the following optimization problem,
\begin{equation}\label{e2}
\min_{\mathcal{P},\mathcal{S}}\|\mathcal{L}\|_*+\lambda\|\mathcal{S}\|_1,
\quad s.t.\, \mathcal{G}\mathcal{(\hat{S}+\hat{L})}={\hat{g}}
\end{equation}

In order to solve this problem by ADMM, we introduce $\mathcal{P}, \mathcal{Q}$ and then obtain the following optimization problem,
\begin{equation}\label{e3}
\min_{\mathcal{S},\mathcal{L},\mathcal{P},\mathcal{Q}}
\|\mathcal{P}\|_*+\lambda\|\mathcal{Q}\|_1+
{\bf1}_{\mathcal{G}\mathcal{(\hat{S}+\hat{L})}={\hat{g}}},
\quad s.t.\, \mathcal{P=L,~Q=S}.
\end{equation}


The Lagrangian is augmented as
\begin{align}
\mathcal{L}_{\rho_1,\rho_2}(\mathcal{S},\mathcal{L},\mathcal{P},\mathcal{Q})
&=\|P\|_*+\lambda \|\mathcal{Q}\|_1\\
&+\frac{\rho_1}{2}\|\mathcal{L-P}+\rho_1^{-1}\mathcal{Z}_1\|_{F}^2\\
&+\frac{\rho_2}{2}\|\mathcal{S-Q}+\rho_2^{-1}\mathcal{Z}_2\|_{F}^2\\
&+\frac{\rho_3}{2}\|G\mathcal{(S(:)+L(:))}-g+\rho_3^{-1}\mathcal{Z}_3\|_{F}^2
\end{align}

Each of the above sub-problems can be solved as the followings,
%\begin{align}
%\mathcal{P}
%&=arg\min_{\mathcal{P}}\frac{1}{2}\|\mathcal{L-P}+\rho_1^{-1}\mathcal{Z}_1\|_F^2
% +\frac{1}{\rho_1}\|P\|_*\notag\\
%&=D_{1/\rho_1}(\mathcal{L}+\rho_1^{-1}\mathcal{Z}_1)\\
%\mathcal{Q}
%&=arg\min_{\mathcal{Q}}\frac{1}{2}\|\mathcal{S-Q}+\rho_2^{-1}\mathcal{Z}_1\|_F^2
% +\frac{\lambda}{\rho_2}\|Q\|_1\notag\\
%&=S_{\lambda/\rho_2}(\mathcal{S}+\rho_2^{-1}\mathcal{Z}_2)\\
%\mathcal{L}
%&=arg\min_{\mathcal{L}}\frac{\mu}{2}\|G\mathcal{(S(:)+L(:))}-g\|_{F}^2
% +\frac{\rho_1}{2}\|\mathcal{L-P}+\rho_1^{-1}\mathcal{Z}_1\|_{F}^2\notag\\
%&=(\mu G^TG+\rho_1I)^{-1}(\mu G^T g+\rho_1\mathcal{P}(:)-\mathcal{Z}_1
% -\mu G^TG\mathcal{S}(:))\\
%\mathcal{S}
%&=arg\min_{\mathcal{S}}\frac{\mu}{2}\|G\mathcal{(S(:)+L(:))}-g\|_{F}^2
% +\frac{\rho_2}{2}\|\mathcal{S-Q}+\rho_2^{-1}\mathcal{Z}_2\|_{F}^2\notag\\
%&=(\mu G^TG+\rho_2I)^{-1}(\mu G^T g+\rho_2\mathcal{Q}(:)-\mathcal{Z}_2
% -\mu G^TG\mathcal{L}(:))
%\end{align}
\begin{align}
\mathcal{P}
&=arg\min_{\mathcal{P}}\frac{1}{2}\|\mathcal{L-P}+\rho_1^{-1}\mathcal{Z}_1\|_F^2
 +\frac{1}{\rho_1}\|P\|_*\notag\\
&=D_{1/\rho_1}(\mathcal{L}+\rho_1^{-1}\mathcal{Z}_1)\\
\mathcal{Q}
&=arg\min_{\mathcal{Q}}\frac{1}{2}\|\mathcal{S-Q}+\rho_2^{-1}\mathcal{Z}_1\|_F^2
 +\frac{\lambda}{\rho_2}\|Q\|_1\notag\\
&=S_{\lambda/\rho_2}(\mathcal{S}+\rho_2^{-1}\mathcal{Z}_2)\\
\mathcal{L}
&=arg\min_{\mathcal{L}}
\frac{\rho_1}{2}\|\mathcal{L-P}+\rho_1^{-1}\mathcal{Z}_1\|_{F}^2+
\frac{\rho_3}{2}\|G\mathcal{(S(:)+L(:))}-g+\rho_3^{-1}\mathcal{Z}_3\|_{F}^2\notag\\
&=(\rho_3 G^TG+\rho_1I)^{-1}
(\rho_3 G^T g+\rho_1\mathcal{P}(:)-\mathcal{Z}_1-\rho_3 G^TG\mathcal{S}(:)-G^TZ_3)\\
\mathcal{S}
&=arg\min_{\mathcal{S}}
\frac{\rho_2}{2}\|\mathcal{S-Q}+\rho_2^{-1}\mathcal{Z}_2\|_{F}^2+
\frac{\rho_3}{2}\|G\mathcal{(S(:)+L(:))}-g+\rho_3^{-1}\mathcal{Z}_3\|_{F}^2\notag\\
&=(\rho_3 G^TG+\rho_1I)^{-1}
(\rho_3 G^T g+\rho_2\mathcal{Q}(:)-\mathcal{Z}_2-\rho_3 G^TG\mathcal{L}(:)-G^TZ_3)
\end{align}

Solve this by ADMM by introducing two dual problems, the algorithm can be shown as follows,
%\begin{algorithm}[!htbp]
%  \caption*{\textbf{Algorithm: Solve TCPCP by ADMM}}
%  \begin{algorithmic}[1]
%    \State Initialization $\mathcal{S}=\mathcal{L}=\mathcal{P}=\mathcal{Q}=\mathcal{Z}_1=\mathcal{Z}_2=zeros(dim(X_0))$
%    \For{$k=1,2,\cdots$}
%        \State update $\mathcal{P}$: $\mathcal{P}=D_{\frac{1}{\rho_1}}(\mathcal{L}+\rho_1^{-1}\mathcal{Z}_1)$
%        \State update $\mathcal{Q}$: $\mathcal{P}=S_{\frac{\lambda}{\rho_2}}(\mathcal{S}+\rho_2^{-1}\mathcal{Z}_2)$
%        \State update $\mathcal{L}$: $\mathcal{L}(:)$ = $(\mu G^TG+\rho_1I)^{-1}
%                                                (\mu G^T g+\rho_1\mathcal{P}(:)-\mathcal{Z}_1-\mu G^TG\mathcal{S}(:))$
%        \State update $\mathcal{S}$: $\mathcal{S}(:)$ = $(\mu G^TG+\rho_1I)^{-1}
%                                                (\mu G^T g+\rho_2\mathcal{Q}(:)-\mathcal{Z}_2-\mu G^TG\mathcal{L}(:))$
%        \State dual update $\mathcal{Z}_1$: $\mathcal{Z}_1=\mathcal{Z}_1+\rho_1*(\mathcal{L-P})$
%        \State dual update $\mathcal{Z}_2$: $\mathcal{Z}_2=\mathcal{Z}_2+\rho_2*(\mathcal{S-Q})$
%    \EndFor
%  \end{algorithmic}
%\end{algorithm}

\begin{algorithm}[!htbp]
  \caption*{\textbf{Algorithm: Solve TCPCP by ADMM}}
  \begin{algorithmic}[1]
    \State Initialization $\mathcal{S}=\mathcal{L}=\mathcal{P}=\mathcal{Q}=\mathcal{Z}_1=\mathcal{Z}_2=zeros(dim(X_0))$
    \For{$k=1,2,\cdots$}
        \State update $\mathcal{P}$: $\mathcal{P}=D_{\frac{1}{\rho_1}}(\mathcal{L}+\rho_1^{-1}\mathcal{Z}_1)$
        \State update $\mathcal{Q}$: $\mathcal{P}=S_{\frac{\lambda}{\rho_2}}(\mathcal{S}+\rho_2^{-1}\mathcal{Z}_2)$
        \State update $\mathcal{L}$: $\mathcal{L}(:)$ = $(\rho_3 G^TG+\rho_1I)^{-1}
                                                (\rho_3 G^T g+\rho_1\mathcal{P}(:)-\mathcal{Z}_1-\rho_3 G^TG\mathcal{S}(:)-G^TZ_3)$
        \State update $\mathcal{S}$: $\mathcal{S}(:)$ = $(\rho_3 G^TG+\rho_1I)^{-1}
                                                (\rho_3 G^T g+\rho_2\mathcal{Q}(:)-\mathcal{Z}_2-\rho_3 G^TG\mathcal{L}(:)-G^TZ_3)$
        \State dual update $\mathcal{Z}_1$: $\mathcal{Z}_1=\mathcal{Z}_1+\rho_1*(\mathcal{L-P})$
        \State dual update $\mathcal{Z}_2$: $\mathcal{Z}_2=\mathcal{Z}_2+\rho_2*(\mathcal{S-Q})$
        \State dual update $\mathcal{Z}_3$: $\mathcal{Z}_3=\mathcal{Z}_3+\rho_3*(G\mathcal{(S(:)+L(:))}-g)$
    \EndFor
  \end{algorithmic}
\end{algorithm}


%%========================文======== 档======== 结========束========================
\newpage    %%如缺此行, 末页页眉将显示异常
\end{CJK*}
\end{document}
